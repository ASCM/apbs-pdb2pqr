%% LyX 1.5.6 created this file.  For more info, see http://www.lyx.org/.
%% Do not edit unless you really know what you are doing.
\documentclass[english,preprint]{revtex4}
\usepackage[T1]{fontenc}
\usepackage[latin9]{inputenc}
\usepackage{graphicx}
\usepackage{babel}

\begin{document}

\title{MATLAB version number 5 of the APBS}


\date{\today}

\begin{abstract}
This solver uses the biconjugate gradient stabilized method and the
inexact LU decomposition to numerically solve the linearized PB equation
on a cartesian 3D-grid. This version requires the shifted dielectric
and the ion accessibility coefficient (kappa function) maps as generated
by the APBS code as well as the corresponding pqr file generated by
the pdb2pqr code. It uses standard three-linear splines (spl0) to
spread the charge density along the nearest grid points if needed.
It is able to solve the linear PB eq with either Dirichlet or focus
boundary conditions. In the later case, this code solves the PB equation
in a large (low resolution) domain and the resulting electrostatic
potential solution is subsequently used to evaluate the Dirichlet
boundary condition to solve the PB equation in a (higher resolution)
sub-domain region. The resulting electrostatic potential and charge
maps for both the coarse and target grids are saved in dx format in
different folders. For visualization purpose, this code also generates
two files (.fig and .tiff) corresponding to the graphical representation
of the electrostatic potential surface. This version was used to solve
the pka example provided by the apbs package. If the Dirichlet Boundary
Condition is required, then this version basically provides the same
results than the previous one. The main difference is that the user
doesn't have to edit the source files in this version but only have
to provide the target input-file name and the corresponding full path
as the only argument of the (new) matlab function MAPBS (x). See the
pka example and the MATLAB\_PB\_SOLVER\_5 package for more details.
\end{abstract}
\maketitle

\subsection*{Description}

This code is based on Michel Holst's thesis and Nathan Baker's APBS
approach. The box-method is used to discretize the following (linearized)
PB equation 

\begin{equation}
-\nabla.\left(\epsilon\left(\mathbf{r}\right)\nabla u\left(\mathbf{r}\right)\right)+\bar{\kappa}\left(\mathbf{r}\right)^{2}u\left(\mathbf{r}\right)=magic\sum_{i=1}^{N}z_{i}\delta\left(\mathbf{r}-\mathbf{r}_{i}\right)\label{eq:one}\end{equation}


where $u\left(\mathbf{r}\right)=e_{c}\Phi\left(\mathbf{r}\right)/K_{B}T$
and $magic=4\pi e_{c}^{2}/K_{B}T.$ For a diagonal dielectric tensor,
the resulting discretized linear PB equations at the nodes $u_{ijk}=u\left(x_{i},y_{j},z_{k}\right)$
for $1\leq i\leq N_{x}$, $1\leq j\leq N_{y}$ and $1\leq k\leq N_{z}$
reads 

\[
\left[\epsilon_{i-1/2,j,k}^{x}\frac{\left(h_{j-1}+h_{j}\right)\left(h_{k-1}+h_{k}\right)}{4h_{i-1}}+\epsilon_{i+1/2,j,k}^{x}\frac{\left(h_{j-1}+h_{j}\right)\left(h_{k-1}+h_{k}\right)}{4h_{i}}+\right.\]


\[
\epsilon_{i,j-1/2,k}^{y}\frac{\left(h_{i-1}+h_{i}\right)\left(h_{k-1}+h_{k}\right)}{4h_{j-1}}+\epsilon_{i,j+1/2,k}^{y}\frac{\left(h_{i-1}+h_{i}\right)\left(h_{k-1}+h_{k}\right)}{4h_{j}}+\]


\[
\epsilon_{i,j,k-1/2}^{k}\frac{\left(h_{i-1}+h_{i}\right)\left(h_{j-1}+h_{j}\right)}{4h_{k-1}}+\epsilon_{i,j,k+1/2}^{k}\frac{\left(h_{i-1}+h_{i}\right)\left(h_{j-1}+h_{j}\right)}{4h_{k}}+\]


\[
\left.\kappa_{ijk}\frac{\left(h_{i-1}+h_{i}\right)\left(h_{j-1}+h_{j}\right)\left(h_{k-1}+h_{k}\right)}{8}\right]u_{ijk}+\]


\[
\left[-\epsilon_{i-1/2,j,k}^{x}\frac{\left(h_{j-1}+h_{j}\right)\left(h_{k-1}+h_{k}\right)}{4h_{i-1}}\right]u_{i-1jk}+\left[-\epsilon_{i+1/2,j,k}^{x}\frac{\left(h_{j-1}+h_{j}\right)\left(h_{k-1}+h_{k}\right)}{4h_{i}}\right]u_{i+1jk}+\]


\[
\left[-\epsilon_{i,j-1/2,k}^{y}\frac{\left(h_{i-1}+h_{i}\right)\left(h_{k-1}+h_{k}\right)}{4h_{j-1}}\right]u_{ij-1k}+\left[-\epsilon_{i,j+1/2,k}^{y}\frac{\left(h_{i-1}+h_{i}\right)\left(h_{k-1}+h_{k}\right)}{4h_{j}}\right]u_{ij+1k}+\]


\[
\left[-\epsilon_{i,j,k-1/2}^{k}\frac{\left(h_{i-1}+h_{i}\right)\left(h_{j-1}+h_{j}\right)}{4h_{k-1}}\right]u_{ijk-1}+\left[-\epsilon_{i,j,k+1/2}^{k}\frac{\left(h_{i-1}+h_{i}\right)\left(h_{j-1}+h_{j}\right)}{4h_{k}}\right]u_{ijk+1}=\]


\begin{equation}
magic\frac{\left(h_{i-1}+h_{i}\right)\left(h_{j-1}+h_{j}\right)\left(h_{k-1}+h_{k}\right)}{8}f_{ijk}\label{eq:two}\end{equation}


in which  

\[
h_{i}=x_{i+1}-x_{i},\: h_{j}=y_{j+1}-y_{j}\: h_{k}=z_{k+1}-z_{k}\]


The delta functions appearing in the right hand side of the starting
equations are approximated with linear B-splines (spl0) which spread
the point like charge along the nearest neighborhood. The resulting
$f_{ijk}$ represent the smearing of the point charges along the grid
points. 

For more details, including used unit system, please refer to the
Michel Holst's thesis and the APBS user guide online. To visualize
more clearly the problem, let's explicitly write the first equations
for a cubic grid of 5x5x5 containing general coefficients

\[
a_{222}u_{222}+a_{122}u_{122}+a_{322}u_{322}+a_{212}u_{212}+a_{232}u_{232}+a_{221}u_{221}+a_{223}u_{223}=f_{222}\]


\[
a_{322}u_{322}+a_{222}u_{222}+a_{422}u_{422}+a_{312}u_{312}+a_{332}u_{332}+a_{321}u_{321}+a_{323}u_{323}=f_{322}\]


\[
a_{422}u_{422}+a_{122}u_{122}+a_{322}u_{322}+a_{212}u_{212}+a_{232}u_{232}+a_{221}u_{221}+a_{223}u_{223}=f_{422}\]


\[
a_{232}u_{232}+a_{132}u_{132}+a_{332}u_{332}+a_{222}u_{222}+a_{242}u_{242}+a_{231}u_{231}+a_{233}u_{233}=f_{232}\]


\[
\ldots\]


in which the nodes are arranged using the natural ordering 

\[
U=[u_{111},u_{211},..,u_{N_{x}11,}u_{121},..,u_{221},u_{321},..,u_{N_{x}21}...,u_{N_{x}N_{y}N_{z}}]^{T}\]


Note that the prescribed values of nodes $u_{1jk},u_{N_{x},j,k},u_{i,1,k},u_{i,N_{y},k},u_{ij1}$
and $u_{ijN_{z}}$ along the faces of the box coming from the Dirichlet
boundary conditions will have their corresponding elements removed
in such a way that only equations for the interior nodes remain. In
other words, we will only consider the following set of unknown nodes

\[
U=[u_{222},u_{322},..,u_{N_{x}-1,22,}u_{232},..,u_{332},u_{432},..,u_{N_{x}-2,32}...,u_{N_{x}-1,N_{y}-1,N_{z}-1}]^{T}\]


in such a way that the previous equations become 

\[
a_{222}u_{222}+a_{322}u_{322}+a_{232}u_{232}+a_{223}u_{223}=f_{222}-a_{122}u_{122}-a_{212}u_{212}-a_{221}u_{221}\equiv b_{222}\]


\[
a_{322}u_{322}+a_{222}u_{222}+a_{422}u_{422}+a_{332}u_{332}+a_{323}u_{323}=f_{322}-a_{312}u_{312}-a_{321}u_{321}\equiv b_{322}\]


\[
a_{422}u_{422}+a_{322}u_{322}+a_{232}u_{232}+a_{223}u_{223}=f_{422}-a_{122}u_{122}-a_{212}u_{212}-a_{221}u_{221}\equiv b_{422}\]


\[
a_{232}u_{232}+a_{332}u_{332}+a_{222}u_{222}+a_{242}u_{242}+a_{233}u_{233}=f_{232}-a_{132}u_{132}-a_{231}u_{231}\equiv b_{232}\]


in which the boundary $u's$ are conveniently brought to the right-hand-side
of the equations. The resulting left-hand side equations can be written
in compact form in term of matrix vector product as follows

\[
Au=b\]


in which

\[
u\left(p\right)=u_{ijk},\qquad b\left(p\right)=b_{ijk},\qquad p=(k-2)(N_{x}-2)(N_{y}-2)+(j-2)(N_{x}-2)+i-1\]


\[
i=2,..,N_{x}-2,\quad j=2,..,N_{y}-2,\quad k=2,..,N_{z}-2\]


%
\begin{figure}
\includegraphics[scale=0.75]{Amatrix}

\caption{A Matrix representation}

\end{figure}


and $A$ is a (seven banded block tri-diagonal form) $(N_{x}-2)(N_{y}-2)(N_{z}-2)$
by $(N_{x}-2)(N_{y}-2)(N_{z}-2)$ squared symmetric positive definite
matrix containing the following nonzero elements (see figure 1):

\begin{itemize}
\item The main diagonal elements
\end{itemize}
\[
d_{0}(p)=\left[\epsilon_{i-1/2,j,k}^{x}\frac{\left(h_{j-1}+h_{j}\right)\left(h_{k-1}+h_{k}\right)}{4h_{i-1}}+\epsilon_{i+1/2,j,k}^{x}\frac{\left(h_{j-1}+h_{j}\right)\left(h_{k-1}+h_{k}\right)}{4h_{i}}+\right.\]


\[
\epsilon_{i,j-1/2,k}^{y}\frac{\left(h_{i-1}+h_{i}\right)\left(h_{k-1}+h_{k}\right)}{4h_{j-1}}+\epsilon_{i,j+1/2,k}^{y}\frac{\left(h_{i-1}+h_{i}\right)\left(h_{k-1}+h_{k}\right)}{4h_{j}}+\]


\[
\epsilon_{i,j,k-1/2}^{k}\frac{\left(h_{i-1}+h_{i}\right)\left(h_{j-1}+h_{j}\right)}{4h_{k-1}}+\epsilon_{i,j,k+1/2}^{k}\frac{\left(h_{i-1}+h_{i}\right)\left(h_{j-1}+h_{j}\right)}{4h_{k}}+\]


\begin{equation}
\left.\kappa_{ijk}\frac{\left(h_{i-1}+h_{i}\right)\left(h_{j-1}+h_{j}\right)\left(h_{k-1}+h_{k}\right)}{8}\right]\label{eq:three}\end{equation}


\begin{itemize}
\item The Next upper band diagonal, which is shifted in one column to the
left from the first column, contains the following elements
\end{itemize}
\begin{equation}
\left[d_{1}(p)=-\epsilon_{i+1/2,j,k}^{x}\frac{\left(h_{j-1}+h_{j}\right)\left(h_{k-1}+h_{k}\right)}{4h_{i}}\right]\label{eq:four}\end{equation}


\begin{itemize}
\item The second upper band diagonal which is shifted $N_{x}-2$ columns
from the first column 
\end{itemize}
\begin{equation}
d_{2}(p)=\left[-\epsilon_{i,j+1/2,k}^{y}\frac{\left(h_{i-1}+h_{i}\right)\left(h_{k-1}+h_{k}\right)}{4h_{j}}\right]\label{eq:five}\end{equation}


\begin{itemize}
\item The third upper band diagonal which is shifted $(N_{x}-2)(N_{y}-2)$
columns from the first column
\end{itemize}
\begin{equation}
d_{3}(p)=\left[-\epsilon_{i,j,k+1/2}^{k}\frac{\left(h_{i-1}+h_{i}\right)\left(h_{j-1}+h_{j}\right)}{4h_{k}}\right]\label{eq:six}\end{equation}


The remaining elements of the upper triangular squared matrix A are
set equal to zero. By symmetry we obtain the lower triagonal elements
of the matrix A. Because the matrix A is sparse and large, we can
implement efficient methods that optimally solve the linear system
for U. Specifically, we use the biconjugate gradient stabilized method
combined with the inexact LU decomposition of the matrix A. Having
the numerical values for the nodes in the interior of the box, we
finally add the previously removed prescribed values along the six
faces to get the solution over the complete set of grid points.


\subsection*{Dirichlet Boundary Condition}

The values of nodes $u_{1jk},u_{N_{x},j,k},u_{i,1,k},u_{i,N_{y},k},u_{ij1}$
and $u_{ijN_{z}}$ along the six faces of the box is set to the values
prescribed by a Debye-H�ckel model for a multiple, non-interacting
spheres with a point charges. The sphere radii are set to the atomic
radii of the biomolecule and the sphere charges are set to the total
charge of the protein. 


\subsection*{Focus Boundary Condition}

Our code uses linear interpolation to obtain the value of the potential
at the six faces of the target box from the value of the potential
obtained at larger domain. 


\subsection*{Computational algorithm for Dirichlet boundary conditions}

\begin{enumerate}
\item The code reads the (target) input file .inm to get the APBS input
files (shifted dielectric coefficients, kappa function and pqr data
file) as well as the number of (target) grid points, the box Lengths,
temperature, and bulk properties (ionic strength and solvent dielectric
coefficient) among other parameters.
\item The center of the grid is evaluated from the corresponding pqr file.
\item By using linear B-splines, the charge density is discretized to get
$f_{ijk}$ for $i=1,..,N_{x},\quad j=1,..,N_{y},\quad k=1,..,N_{z}$.
The Dirichlet boundary condition along the six faces of the box $u_{1jk},u_{N_{x},j,k},u_{i,1,k},u_{i,N_{y},k},u_{ij1}$
and $u_{ijN_{z}}$ are calculated by using the temperature, the value
of the bulk dielectric coefficient (usually water) and ionic strength.
\item The nonzero components of the matrix $A$, e.g., the diagonal elements
$d_{0}(p),d_{1}(p),d_{2}(p),$ and $d_{3}(p),$ for $p=(k-2)(N_{x}-2)(N_{y}-2)+(j-2)(N_{x}-2)+i-1$
and $i=2,..,N_{x}-1,\quad j=2,..,N_{y}-1,\quad k=2,..,N_{z}-1$ are
evaluated by using the expressions (\ref{eq:three}), (\ref{eq:four}),(\ref{eq:five}),
and (\ref{eq:six}). The values for the shifted dielectric coefficients
and kappa function elements are obtained from the APBS input files.
The values of the mesh size $h_{i},h_{j}$ and $h_{k}$ are obtained
from the number of grid points and the Length of the box. Next, the
sparse upper triangular matrix A is constructed by filling with zeros
the remaining elements of the matrix A. Next, the lower triangular
elements of the matrix A are obtained by using the following symmetry
property $A_{pq}=A_{qp}$ for $q=1,..,(N_{x}-2)(N_{y}-2)(N_{z}-2)$
and $p=q,..,(N_{x}-2)(N_{y}-2)(N_{z}-2)$. 
\item The elements of $b_{ijk}$ are evaluated by using the values obtained
for the discretized charge density $f_{ijk}$ and the values of the
Dirichlet boundary elements multiplied by the appropriate shifted
dielectric coefficient values. The natural ordering $p=(k-2)(N_{x}-2)(N_{y}-2)+(j-2)(N_{x}-2)+i-1$
and $i=2,..,N_{x}-1,\quad j=2,..,N_{y}-1,\quad k=2,..,N_{z}-1$ is
used to construct the corresponding vector $b(p)$ (one index) from
the data array structure (three indices) $b_{ijk}$.
\item The inexact $LU$ decomposition of the matrix $A$ is performed. The
default tolerance value is set equal to 0.25 which provides a fast
evaluation of the matrices $L$ and $U$. 
\item The resulting $L$ and $U$ matrices, the matrix $A$ and the vector
$b$ are used to approximately solve $Au=b$ for the vector $u$ using
the biconjugate gradient stabilized method. The default accuracy is
set equal to 10\textasciicircum{}-9 and the maximum number of iteration
equal to 800. 
\item The natural ordering relationship is used to convert the resulting
vector $u(p)$ to data array structure to get the numerical solution
for $u_{ijk}$ for $i=2,..,N_{x}-1,\quad j=2,..,N_{y}-1,\quad k=2,..,N_{z}-1$.
\item Finally the previously removed values of the nodes at the faces of
the box are used to obtain the solution for the nodes $u_{ijk}$ over
the complete set of grid points, namely for $i=1,..,N_{x},\quad j=1,..,N_{y},\quad k=1,..,N_{z}$.
\item The electrostatic potential $u_{ijk}$ and the charge $f_{ijk}$ maps
are saved in dx format files.
\item The electrostatic potential surface $u_{ij(N_{z}+1)/2}$ is saved
in tiff and fig format files for visualization purpose. 
\end{enumerate}

\subsection*{Computational algorithm for Focus boundary conditions}

The algorithm reads the target input file finding that the boundary
condition line says {}``focusname.inm'' instead of {}``sdh''.
Then the matlab code automatically first reads that input file {}``focusname.inm''
to solve the PB equation in the specified coarse grid using Dirichlet
boundary condition as explained previously. It saves the resulting
electrostatic potential solution in a temporary dx formatted file
and then perform the following steps; 

\begin{enumerate}
\item The code reads the (target) input file .inm to get the APBS input
files (shifted dielectric coefficients, kappa function and pqr data
file) as well as the number of (target) grid points, the box Lengths,
temperature, and bulk properties (ionic strength and solvent dielectric
coefficient) among other parameters.
\item The center of the grid is evaluated from the corresponding pqr file.
\item By using linear B-splines, the charge density is discretized to get
$f_{ijk}$ for $i=1,..,N_{x},\quad j=1,..,N_{y},\quad k=1,..,N_{z}$.
The Dirichlet boundary condition along the six faces of the target
(smaller) box $u_{1jk},u_{N_{x},j,k},u_{i,1,k},u_{i,N_{y},k},u_{ij1}$
and $u_{ijN_{z}}$ are calculated by using a three-linear interpolation
for the electrostatic potential solution obtained previously at larger
boxsides (Focus boundary Condition).
\item The nonzero components of the matrix $A$, e.g., the diagonal elements
$d_{0}(p),d_{1}(p),d_{2}(p),$ and $d_{3}(p),$ for $p=(k-2)(N_{x}-2)(N_{y}-2)+(j-2)(N_{x}-2)+i-1$
and $i=2,..,N_{x}-1,\quad j=2,..,N_{y}-1,\quad k=2,..,N_{z}-1$ are
evaluated by using the expressions (\ref{eq:three}), (\ref{eq:four}),(\ref{eq:five}),
and (\ref{eq:six}). The values for the shifted dielectric coefficients
and kappa function elements are obtained from the APBS input files.
The values of the mesh size $h_{i},h_{j}$ and $h_{k}$ are obtained
from the number of grid points and the Length of the box. Next, the
sparse upper triangular matrix A is constructed by filling with zeros
the remaining elements of the matrix A. Next, the lower triangular
elements of the matrix A are obtained by using the following symmetry
property $A_{pq}=A_{qp}$ for $q=1,..,(N_{x}-2)(N_{y}-2)(N_{z}-2)$
and $p=q,..,(N_{x}-2)(N_{y}-2)(N_{z}-2)$. 
\item The elements of $b_{ijk}$ are evaluated by using the values obtained
for the discretized charge density $f_{ijk}$ and the values of the
Dirichlet boundary elements multiplied by the appropriate shifted
dielectric coefficient values. The natural ordering $p=(k-2)(N_{x}-2)(N_{y}-2)+(j-2)(N_{x}-2)+i-1$
and $i=2,..,N_{x}-1,\quad j=2,..,N_{y}-1,\quad k=2,..,N_{z}-1$ is
used to construct the corresponding vector $b(p)$ (one index) from
the data array structure (three indices) $b_{ijk}$.
\item The inexact $LU$ decomposition of the matrix $A$ is performed. The
default tolerance value is set equal to 0.25 which provides a fast
evaluation of the matrices $L$ and $U$. 
\item The resulting $L$ and $U$ matrices, the matrix $A$ and the vector
$b$ are used to approximately solve $Au=b$ for the vector $u$ using
the biconjugate gradient stabilized method. The default accuracy is
set equal to 10\textasciicircum{}-9 and the maximum number of iteration
equal to 800. 
\item The natural ordering relationship is used to convert the resulting
vector $u(p)$ to data array structure to get the numerical solution
for $u_{ijk}$ for $i=2,..,N_{x}-1,\quad j=2,..,N_{y}-1,\quad k=2,..,N_{z}-1$.
\item Finally the previously removed values of the nodes at the faces of
the box are used to obtain the solution for the nodes $u_{ijk}$ over
the complete set of grid points, namely for $i=1,..,N_{x},\quad j=1,..,N_{y},\quad k=1,..,N_{z}$.
\item The electrostatic potential $u_{ijk}$ and the charge $f_{ijk}$ maps
are saved in dx format files.
\item The electrostatic potential surface $u_{ij(N_{z}+1)/2}$ is saved
in tiff and fig format files for visualization purpose. 
\end{enumerate}
\emph{Comment1:} Note that in this case the user have to provide two
inm.files and the corresponding dx and pqr files for both the coarse
and target grids. 

\emph{Comments2:} In this version the user have to provide two pqr
files, one representing the molecule by which the PB eq is solved
and, the second one to define the center of the grid. It may be the
same than the first one, but in general, for complex systems they
are not. 

\emph{Comments3:} Fianlly, the user have to provide both directories
for the input and output files respectively. In this way the user
doesn't have to edit the source files at all. Just need to provide
the name of the input file and the full path as the only argument
in the Matlab function MPABS (x). 
\end{document}
